\chapter{Proposed solution}
\label{chapter:proposedSolution}

\section{Approach definition}
\label{section:approachDefinition}


\section{Technologies used}
\label{section:technologiesUsed}


\subsection{The Spring Framework}
\label{subsection:springBootFramework}

% https://blog.eduonix.com/java-programming-2/learn-design-patterns-used-spring-framework/

\subsection{Spring Boot}
\label{subsection:springBoot}

\subsection{Spring Security}
\label{subsection:proposedSolution}

\subsection{Spring JDBC}
\label{subsection:springJDBC}

\subsection{ReactJS Framework}
\label{section:reactJSFramework}

\subsection{Redux}
\label{section:redux}

\subsection{MySQL Database}
\label{section:mysqlDatabase}

% - performance (big amount of data)
% - relational vs non-relational db
% - where to store files? (db vs cloud)
% - db stuff (indexing / speeding up things ? highPerformanceMySQL)
% There's usually a lot of data, a moderate system will have over 1GB of data
% organized in tens of millions of records, so much data that managing it is a major part
% of the system. Older systems used indexed file structure such as IBM's VSAM and
% ISAM. Modern systems usually use databases, mostly relational databases. The
% design and feeding of these databases has turned into a sub-profession of its own.




% just so items from .bib would show up in bibliography
\cite{buildingRESTfulWebServicesWithSpring}
\cite{tamingTheStateInReact}
\cite{highPerformanceMySQL}


