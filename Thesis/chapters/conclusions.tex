\chapter{Conclusions}
\label{chapter:conclusions}

The application that is the result of the research and development conducted in this thesis successfully manages to address the requirements regarding the document management process. It represents a good alternative to the current flow, minimizing human effort and need for interaction by automating parts of the process. In addition, the system provides great features for tracking documents and their status. This makes it easier to identify which status a particular document has at any given time, but it also allows to manage the register of documents as a whole, quickly and efficiently getting all the desired information and statistics from the app.

Designing and implementing the Registry System had its own challenges. We had to make the app secure so that users would trust their data with it. We had to keep performance in mind while implementing the application logic. Last but not least, we had to design the application in a way that offers a good user experience. After all, regardless of the complex logic it provides, an app is not going to raise efficiency if the user is frustrated and confused using it. That is why we tried making all the features as intuitive and easy to learn as possible.

As a next step, we plan to deploy our Registry System Application, either on a physical server or a cloud platform, after which the application would be ready to be used by the opera employees.

The main achievement of this thesis is that it is meant to digitalize an existing process in a public institution, which, if we look at the bigger picture, is a small piece in the digitalization and modernization of the Cluj-Napoca city.
