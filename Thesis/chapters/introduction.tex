\chapter{Introduction}
\label{chapter:introduction}

\section{Context}
\label{section:context}

An institution like the National Romanian Opera of Cluj-Napoca keeps track of a lot of documents.
Employees from different departments including but not limited to the economic, marketing or human resources departments, send and receive documents as part of their work on a daily basis. The institution should track each single incoming and outgoing document, identify each one by an unique registry number and know whether it has reached its recipient as well as the status of the document.

Currently, for achieving this task, the institution maintains a physical register in the form of a catalog. Each document has its corresponding row in the register which was manually added at the moment of registration. This takes time and effort which could be spared by using an automated solution.

\section{Motivation}
\label{section:motivation}

A lot of documents means plenty of information, and working with a huge amount of information which is only stored on paper is usually a difficult and tedious task. This could be simplified by introducing a digital solution into the document management process, which would considerably improve its efficiency. In addition, manual document administration is prone to human errors, which could be avoided by automating some parts of the process.



\section{Original contributions}
\label{section:originalContributions}

The main contribution of this thesis is designing and developing a document management application which would be used by the National Romanian Opera of Cluj-Napoca. This includes a considerable amount of research related to all stages of a software product lifecycle. As part of the the requirements engineering process, discussions with the Opera employees were conducted to get first hand information regarding their expectations for the application. Another big part of the research was related to software architecture and technologies. This was needed because choosing the best approach is relevant when developing an application that would ultimately be used by a real client. Last but not least, the implementation of the app had to take into consideration security, performance and usability, to make interacting with the app both efficient and intuitive.

\section{Structure of the thesis}
\label{section:structureOfTheThesis}


