\chapter{State of the Art}
\label{chapter:stateOfTheArt}

Researching the existence of similar applications has proven that online document management systems are quite popular among both public institutions and private companies. Those who already embraced the digitalization of their document archives affirm that it helped keeping documents more organized and significantly improved search effectiveness. In addition, such systems offer significant advantages in terms of information retrieval, security and lower cost of operations.

A fair amount of research has been conducted on designing and developing web-based distributed administration system services. The article "Optimization for Web-based Online Document Management" \cite{Cheng-2013} describes in detail a "web-based document life-cycle management model" which handles documents from an institute library beginning with their creation till reaching archived state. It focuses on effectiveness and real-time tracking of the document workflow routine, constant availability of the service regardless of network delays, version control and archivation of the documents using a cloud-based solution.

While some institutions choose to develop their own digital document management platforms, others turn to already implemented solutions. A good example of such a solution is Regista - "A complete application for registry and document management" \cite{regista}. With more than 500 different clients in 40 romanian counties, it has a total of more than 7 million documents registered through their application. It offers features like automating the handling of the register of incoming and outgoing documents, classification and distribution of documents, all of which should considerably improve time efficiency. In addition, it facilitates information sharing between the document issuer and recipient, simplifying the way requests are made and responses are emitted, which again speeds up the process of solving document related tasks. Their clients count a broad spectrum of institutions from the industry, education, public administration and healthcare systems among many others, which only demonstrates that the digitalization of the document management process is a growing trend in various fields.


